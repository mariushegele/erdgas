
\section{Ist Erdgas nachhaltig?}

Die EU-Taxonomie für nachhaltige Entwicklung soll ein Goldstandard für grüne Investitionen sein und dadurch klimafreundliche Investitionen in die Projekte lenken, die wirklich dabei helfen werden den Planeten langfristig zu schützen \cite{reuters}. Am 31.12.2021 fasste die EU-Kommision einen Beschlussentwurf nach dem Atom- und Gaskraftwerke als grüne Technologie in die Taxonomie eingehen sollen \cite{taz-taxonomie}. Das soll gelten wenn ein Gaskraftwerk eine Altanlage (Kohlekraftwerk) ersetzt \cite{eu-komm} und wenn sie maximal 270 g/kWh direkte Emissionen oder 550 g/kW durchschnittliche Emisssionen über 20 Jahre produziert \cite{uba}. Letzteres würde 1,4 Millarden Tonnen CO2-Äquivalente als nachhaltig deklarieren, wenn alle bestehenden Kohlekraftwerke durch Gaskraftwerke ersetzt werden würden. Dazu kommt, dass die letztere Anforderung keinen Schwellenwert für Emissionen pro kWh definiert und damit die jährlich Treibhausgasemissionen aus Gaskraftwerken schwer bestimmbar macht.

In einer Ausschusssitzung haben sich EU-Parlamentarier am 14.06.2022 gegen diesen Beschlussentwurf ausgesprochen \cite{spiegel-taxonomie}. Damit wird der Vorschlag wahrscheinlich im Juli 2022 kippen. In diesem Abschnitt soll dennoch darauf eingegangen werden mit welchem Argumenten die Position der EU-Kommision vertreten wird und was dagegen spricht Gaskraftwerke als nachhaltige Investitionen zu deklarieren. 

Ein weiterer aktueller Streitpunkt ist der Bau der Süddeutsche Erdgasleitung (SEL) \cite{terranets}. Diese soll über 250 km von Lampertheim in Hessen durch Baden-Württemberg nach Bissingen in Bayern fließen. Die bauende terranets BW beschreibt sie als kluge Investition in die Zukunft, die die Region über die nächsen Jahre mit dringend benötigtem Erdgas versorgt. Sie haben eine Befragung bei Verteilnetzbetreibern, Stadtwerken, Kraftwerken und der Industrie gemacht und daraus bestimmt, dass der Gasbedarf von 39,5 GW 2022 auf 49,1 GW 2032 wachsen wird.
Dieser Abschnitt wird neben der von der EU-Komissionen eingeschätzten Nachhaltigkeit von Gaskraftwerken auch die Nachhaltigkeit von Investitionen in das Gasnetz einschätzen.

\begin{figure}
\centering
\includegraphics[width=12cm]{../pres/fig/multimodal-energy-grid.png}
\caption{Power-to-Gas and Power-to-X: ein multimodales Energienetz \cite{multimodal-grids}}
\label{fig:multimodal-grids}
\end{figure}

In \autoref{sec:Stromproduktion} wurden bereits verschiedene Aspekte von Gaskraftwerken beschrieben: sie sind effizient, haben niedrigere Treibhausgasemissionen als Öl- und Kohlekraftwerke und dienen als Flexibilitätsquelle für Stromnetzbetreiber bei der Frequenzstabilisierung.
Ihre Rolle wird deswegen häufig und so auch von der EU-Kommision als Brückentechnologie beim Kohleausstieg und im Übergang zu erhöhter Stromproduktion aus erneuerbaren Quellen gesehen \cite{reuters, eu-komm}.
Der Energiebedarfs soll kontinuierlich und zuverlässig gedeckt werden \cite{uba}.
Gaskraftwerke können so ausgelegt sein, dass sie auch in der Lage sind Wasserstoff zurückzuverstromen (H2-ready) \cite{rnd}. In diesem Fall können sie auch langfristig Teil der Wasserstoffspeicherinfrastruktur sein. 

Beim Neubau von Gasnetzen kann darauf geachtet werden, dass diese auch bereit sind, Wasserstoff zu transportieren. Aus technischer Sicht gibt es drei Möglichkeiten. Man kann Wasserstoff einem normalen Gasnetz beimischen. Dabei sind 5\% im Fernnetz und 20\% im Verteilnetz denkbar \cite{iis}. Die zweite Möglichkeit ist die Methanisierung von Wasserstoff. Das HELMETH-Verfahren kombiniert die Elektrolyse mit der Methanisierung, bindet dabei CO2, produziert Methan und Wasser und hat einen Wirkungsgrad von 76\% \cite{helmeth}. Das Methan kann dann dem Gasnetz dann unbegrenzt beigeführt werden. Die dritte Möglichkeit besteht darin reine Wasserstoffenetze zu betreiben. Für große Abnehmer wie die Stahl- und Chemieindustrie kann das sinnvoll sein \cite{iis}. Ein reines Wasserstoffnetz erfordert ein anderes Design für Verdichter. Die Umstellung des existierenden Netzes auf ein reines Wasserstoffnetz würde also Kosten mit sich bringen. Mit Blick auf die energetisch äquivalente Alternative der Methanisierung ist eine solche Umstellung also nicht unbedingt sinnvoll.
Allgemeine Vorteile der Übertragung von Energie über das Gasnetz statt dem Stromnetz ist, dass eine große Pipeline mit Durchmesser von einem Meter und 80-100 bar mit 24 GW etwa acht mal so viel Energie übertragen kann wie eine Hochspannungsleitung \cite{iis}. Zusätzlich hat das Gasnetz wie angesprochen bereit jetzt eine sehr große Speicherkapazität.

In Summe zeigt \autoref{fig:multimodal-grids} ein multimodales Energienetz, das die langfristige Speicherung von Energie und Versorgung der Industrie durch die Elektrolyse und Methanisierung sowie Rückverstromung über Gaskraftwerke ermöglicht.

Die für den Transport und die Verstromung von Erdgas gebaute Infrastruktur kann also auch zukünftig Verwendung finden. Aus diesen Gründen könnte man den Schluss ziehen, dass auch die Investition in diese Technik als nachhaltig klassifiziert werden sollte. Doch es gibt einige Argumente, die dagegen sprechen.

Zuallererst ist das Verbrennen von Erdgas (nicht von grünem Wasserstoff) in Gaskraftwerken alles andere als nachhaltig: es produziert Treibhausgase und verwendet einen nicht nachwachsenden Rohstoff. Dazu kommt, dass der Erdgaspipelines Leckagen haben über die Erdgas direkt in die Atmosphäre austritt \cite{reuters}. Das ist ein großes Problem, weil Methan mit 25 CO2-Äquivalenten kurzfristig ein sehr potentes Treibhausgas ist \cite{uba-co2e}.
Eine Klassifikation von Gaskraftwerken als nachhaltige Technologie setzt möglicherweise in Zeiten in denen der Ausbau der Erneuerbaren stark vorangetrieben werden musss, finanziell die falschen Anreize \cite{uba}. Darüber hinaus verstoßen die von der EU-Kommision gegebenen Kriterien für deren Nachhaltigkeit gegen den Grundsatz der Technologieneutralität -- für andere Kraftwerke gelten maximal 100 gCO2e/kWh \cite{uba}.
Auf psychologischer Ebene und im internationalen Vergleich gefährdet diese Klassifikation die Glaubwürdigkeit der Taxonomie und damit die Glaubwürdigkeit der europäischen Anstrengungen im Klimaschutz \cite{dnr}.

Vorangegangene Argumente sprechen aber durchaus dafür die Gasinfrastruktur zu erhalten oder gar auszubauen. Soll das finanziell incentiviert werden, sollte man dies statt in der EU-Taxonomie separat regulieren, jährliche Grenzwerte mit abnehmendem Verlauf für den Einsatz von fossilem Gas definieren und feste Kriterien für den Einsatz von kohlenstoffarmen Gas definieren \cite{uba}.

