%% LaTeX2e class for seminar theses
%% sections/abstract_en.tex
%% 
%% Karlsruhe Institute of Technology
%% Institute for Program Structures and Data Organization
%% Chair for Software Design and Quality (SDQ)
%%
%% Dr.-Ing. Erik Burger
%% burger@kit.edu

Der russische Angriffskrieg und die Klimakrise motivieren die Reduktion des deutschen Gasverbrauchs und der Importe aus Russland. Ist es möglich das russische Gas durch anderes Gas zu ersetzen? Macht man sich damit nur abhängig von anderen Autokratien? Gibt es Alternativen im Verbrauch? Die Abhängigkeit muss aus dem Sichtpunkt unterschiedlicher Anwendungsbereiche -- Wärme, Strom und Industrie -- betrachtet werden. 

Die EU-Kommision schlägt vor Erdgas und Atomkraft in der EU-Taxonomie zur nachhaltigen Entwicklung als nachhaltig einzustufen. Mit welchen Argumenten wird diese Entscheidung begründet? Was ist dem entgegenzuhalten?